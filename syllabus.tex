\documentclass[11pt,letterpaper,oneside]{article}
\usepackage[latin1]{inputenc}
\usepackage{amsmath}
\usepackage{amsfonts}
\usepackage{amssymb}
\usepackage{graphicx}
\usepackage{hyperref}
\usepackage[margin=.65in]{geometry}
\usepackage{multirow}
\usepackage[compact,small]{titlesec}
\title{\bf \large \vspace{-.5in} Introduction to Bioinformatics Programming in Python\\MSCBIO 2025 - FALL 2021 (8/27 - 12/17)}
\author{}
\date{}
\begin{document}

\maketitle
\vspace{-.5in}
\thispagestyle{empty}
\vspace{-.25in}

\paragraph*{Instructor:} 
\begin{tabular}[t]{ll}
 & David Koes \\
 & 748 Murdoch Building\\
% & Department of Computational \& Systems Biology \\
%  & University of Pittsburgh \\
  & \texttt{dkoes@pitt.edu} \\
\end{tabular}

%\paragraph*{Teaching Assistant:} Hannah Schriever (\texttt{hcs31@pitt.edu})
\paragraph*{Location:} Murdoch 814
\paragraph*{Website:} \url{http://mscbio2025.net/}
\paragraph*{Class Discussion and Announcements:} \url{http://mscbio2025.slack.com}
\paragraph*{Zoom Link:} \url{https://pitt.zoom.us/j/95735576127}
\paragraph*{Optional Books:} {\it Bioinformatics Programming Using Python}.  ISBN: 978-0-596-15450-9. \\
{\it Data Science from Scratch: First Principles with Python}. ISBN: 978-1491901427.

\paragraph*{Course Description:}
The course will introduce students to a variety of computational tools for solving common problems in biological research. Students will be taught the Python programming language through hands on exercises and assignments. Students will acquire knowledge and programming skills that will increase their productivity as researchers.

\paragraph*{Lectures and Recitations:}  
There will be two lectures a week, \textbf{Tuesday and Thursdays from 12:00pm to 1:20pm}. Lectures will be offered both in-person in Murdoch 814 classroom and over Zoom.  Students should choose the lecture modality they are most comfortable with and that maximizes their learning experience. \textbf{All lectures are recorded and are accessible to anyone with a Pitt account and the Panopto link.} The class will start with a short series of lectures that cover basic Linux and Python concepts. These will be followed by lectures that use these basic concepts in a hands-on interactive lab exercise that focuses on a single problem domain and programming toolkit. 

\paragraph*{Communication}
The schedule and assignments will be posted to the class webpage. We will use Slack for class discussion and virtual office hours.  When possible, students should prefer group discussions in resolving problems rather than direct messaging and email.


\paragraph*{Assignments}  There will be 12 programming assignments and a final project. Each assignment will accomplish a common and useful task related to research in computational biology. Most assignments will allow for partial credit.  Assignments will be auto-graded.  
You may submit your assignment as many times as you wish before the deadline until you achieve a working submission.  

\paragraph*{Late Policy} Work submitted up to 1 day late will receive 90\% credit.  Work submitted up to 2 days late will receive 50\% credit.  The late penalty is only applied to additional points earned by the late submission.  For example, if the on-time submission earned an 80\% and a submission 25 hours after the deadline earned a 100\%, the final grade would be 90\%. Requests for extensions should be rare, well-justified and made in advance if possible.

\paragraph*{Grades}  Course grades will be determined completely by the successful and timely completion of assignments with each assignment contributing to 5--10\% 
of the total grade and the project 15\%.  The exact weighting of the assignments and project will be assigned at the end of the course at the discretion of the instructor. The cutoff for an A will be 93\% and 85\% for a B.

\paragraph*{Academic Honesty}  You must do all your own work.  You are encouraged to discuss general concepts, strategies for debugging, and the particulars of a specific software package with other class members.  However, specifics of individual assignments should not be discussed, and you should not show your code to fellow classmates.  Any attempt to `hack' the autograder will result in expulsion from the class and a referral to the dean's office.

\paragraph*{COVID19}  All University policies will be followed and will take precedence over any course policies.  See \url{https://www.coronavirus.pitt.edu/} for more information.    Students should make every reasonable effort to attend class in real-time as in-class group work and discussion is an important part of lecture.  However, all lectures will be recorded and available for asynchronous viewing.


\end{document}
